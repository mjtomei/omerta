% Abstract section - shared between whitepaper and full paper

\begin{abstract}
Decentralized compute sharing faces a fundamental challenge: how can strangers cooperate without a trusted central authority? Blockchain-based systems address this through proof-of-work or proof-of-stake consensus, achieving Byzantine fault tolerance at the cost of significant resource overhead and limited throughput. We present Omerta, a practical implementation that synthesizes decades of research on trust, reputation, and mechanism design into a working system for decentralized compute markets.

Omerta computes trust \emph{locally} relative to each observer, rather than maintaining global scores. Trust is derived from verified transactions---actual compute sessions with measurable outcomes---rather than subjective ratings. This design is \emph{machine-native}: unlike prior reputation systems that assumed humans would rate each other, Omerta expects trust signals to be generated automatically from transaction outcomes, enabling measurement at scales human rating could never achieve. This enables natural scaling without global consensus overhead, at the cost of accepting that different observers may have different views of the same identity's trustworthiness. Crucially, local computation does not mean isolated computation: transactions are witnessed by third parties, records propagate through gossip, and cryptographic signatures ensure authenticity. We retain the benefits of distributed verification and immutable records---we simply don't require the entire network to agree on a single global ordering before transactions can proceed.

This paper presents: (1) a trust model derived from verified transactions rather than subjective ratings; (2) local trust computation with path decay; (3) automated monetary policy that adjusts parameters in response to detected threats; (4) economic analysis demonstrating when unreliable home compute creates genuine value; and (5) double-spend resolution where currency ``weight'' scales with network performance, providing a practical alternative to global consensus for bilateral transactions.

We draw on the observation that human societies have always traded privacy for trust---villages had high trust precisely because everyone knew everyone's business. Omerta recreates this visibility at global scale through on-chain transparency. Unlike villages with their arbitrary social punishment, we aim to maximize freedom within the trust constraint: only verifiable anti-social behavior---failed deliveries, double-spends, verification failures---affects trust scores, where ``verifiable'' means the determination is reproducible from on-chain data by any observer.

We argue that implementing fair trust systems at scale was computationally intractable until machine intelligence provided the reasoning capacity to model behavior, tune parameters, and explain decisions. This paper itself was developed through human-machine collaboration, demonstrating the thesis: machine intelligence both demands the compute that systems like Omerta could provide and enables the trust mechanisms that make such systems work.

Prior trust systems like EigenTrust and FIRE were never widely deployed---they remained academic exercises, computing trust scores that connected to nothing. Omerta brings these ideas into implementation with real economic consequences: trust scores that affect payments, transfer costs, and access. The contribution is both theoretical and practical---new mechanisms where prior work was insufficient, adoption of proven approaches where they exist, and integration into a working system. The software is given away for free with no preallocation of tokens.

Unlike prior decentralized compute platforms that struggled with adoption, Omerta targets a specific opportunity: billions of home computers sit idle most of the time, representing low marginal cost compute for owners who have already paid for hardware and internet. The software will provide transparency about actual operating costs---electricity, bandwidth, wear---with user controls for participation thresholds. The software is open source with no platform fees. And machine intelligence dramatically increases the utility of distributed compute---enabling humans to orchestrate complex parallel workloads across unreliable infrastructure in ways they couldn't manage manually. This combination of low-cost supply, zero-rent platform, and machine-intelligence-amplified demand may succeed where blockchain-based alternatives with mining overhead, token economics, and human-centric design have struggled.
\end{abstract}
