% Section 5: Economic Mechanisms

\section{Economic Mechanisms}
\label{sec:economic-mechanisms}

Prior trust systems---EigenTrust, TidalTrust, FIRE, PowerTrust---computed reputation scores that had no direct consequences. A user with high trust and a user with low trust experienced the same platform. This may explain why these systems remained academic: reputation without economic consequence is reputation without purpose.

Omerta integrates trust directly into economic flows. Trust affects what you earn, what you can transfer, and what you receive from daily distribution. This gives trust a reason to exist and gives participants a reason to care about it.

\subsection{Payment Splits}
\label{subsec:payment-splits}

Consumer payments split between provider and cryptographic burn based on provider trust:

\begin{equation}
\text{provider\_share} = 1 - \frac{1}{1 + K_{\text{PAYMENT}} \times T}
\end{equation}

As trust increases, provider share approaches 100\% asymptotically but never reaches it. New providers with low trust see most of their payment burned; established providers keep most of it.

\subsubsection{Payment Split by Trust Level}

The following table shows provider share for different trust scores and $K_{\text{PAYMENT}}$ values:

\begin{table}[H]
\centering
\begin{tabular}{@{}lrrrr@{}}
\toprule
\textbf{Trust Score} & \textbf{K=1.0} & \textbf{K=2.0} & \textbf{K=5.0} & \textbf{K=10.0} \\
\midrule
0.0 (new) & 0\% & 0\% & 0\% & 0\% \\
0.1 & 9\% & 17\% & 33\% & 50\% \\
0.2 & 17\% & 29\% & 50\% & 67\% \\
0.5 & 33\% & 50\% & 71\% & 83\% \\
1.0 & 50\% & 67\% & 83\% & 91\% \\
2.0 & 67\% & 80\% & 91\% & 95\% \\
5.0 & 83\% & 91\% & 96\% & 98\% \\
10.0 & 91\% & 95\% & 98\% & 99\% \\
\bottomrule
\end{tabular}
\caption{Provider share by trust score and $K_{\text{PAYMENT}}$ parameter}
\label{tab:payment-split}
\end{table}

\textbf{Reading the table}: With $K_{\text{PAYMENT}}=5.0$, a provider with trust score 0.5 keeps 71\% of payments; the remaining 29\% is burned. A highly trusted provider (score 5.0) keeps 96\%.

\textbf{Design insight}: Higher $K$ values reward trust more steeply. $K=5.0$ means a provider must reach trust $\sim$1.0 to keep most of their earnings, creating strong incentive for sustained good behavior.

This creates natural incentives: build trust to keep more of your earnings. No external enforcement needed---the economics handle it.

\subsection{Transfer Burns}
\label{subsec:transfer-burns}

Transfers between identities are taxed based on the minimum trust of sender and receiver:

\begin{equation}
\text{transfer\_burn\_rate} = \frac{1}{1 + K_{\text{TRANSFER}} \times \min(T_{\text{sender}}, T_{\text{receiver}})}
\end{equation}

Low-trust identities cannot easily transfer coins. This prevents reputation laundering (build trust, exploit, transfer coins to fresh identity) and creates strong incentive to donate compute rather than buy coins.

\subsection{Daily Distribution}
\label{subsec:daily-distribution}

New coins are minted daily and distributed proportionally to trust scores:

\begin{equation}
\text{daily\_share}(i) = \frac{\text{effective\_trust}(i)}{\sum_j \text{effective\_trust}(j)} \times \text{DAILY\_MINT}
\end{equation}

This creates the core incentive: misbehave and your trust drops, reducing tomorrow's share. The motivation to be honest is simply: keep getting your handout.

\subsection{Donation and Negative Bids}
\label{subsec:donation-negative-bids}

Providers can accept negative-price bids from research organizations, burning their own coins for accelerated trust:

\begin{table}[H]
\centering
\begin{tabular}{@{}llc@{}}
\toprule
\textbf{Bid Type} & \textbf{Price} & \textbf{Trust Multiplier} \\
\midrule
Commercial & Positive & 1x \\
Zero donation & Zero & 1x \\
Negative donation & Negative & Up to 4x \\
\bottomrule
\end{tabular}
\caption{Trust multipliers by bid type}
\label{tab:bid-types}
\end{table}

This enables bootstrapping: new providers can burn coins to accelerate trust building while providing verified compute to research projects. Trust cannot be purchased without actual work---the burn is additional, not replacement.
